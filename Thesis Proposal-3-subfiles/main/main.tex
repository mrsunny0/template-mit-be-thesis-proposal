%---------------------------------------------------------------------------%
%								   Preamble									%
%---------------------------------------------------------------------------%
% declare document class, 12pt, lettersize and article 
% could also be report, however section headers turn into chapters
\documentclass[12pt, lettersize]{article}

% use subfiles to modularize pieces of the LaTeX code
\usepackage{subfiles}

% import preamble.sty for packages
% import refextdoc.sty for subfile crossreferencing
% note the relative import. Because subfiles (e.g. abstract, introduction, etc.)
% are located in separate files, for all files to be obtainable by main and subfiles,
% need to tranverse up to the root directory (../) then back to the appropriate files in /main
\usepackage{../main/preamble}
\usepackage{../main/refextdoc}

% relative imports of images
% note that graphics path should be encased in {}, ends with / to define directory, 
% and finally separated by , without any lead or lag spaces.
% therefore graphicsspath should look like \graphiscpath{{abc/},{xyz/},{123/}}
% spaces in directory names are not recommended, however the the \usepackage[space]{grffile}
% will attempt to work with directories with spaces
\graphicspath{{../images/}}

%---------------------------------------------------------------------------%
%								Begin Document								%
%---------------------------------------------------------------------------%
\begin{document}

%------------------
% Table of Contents
%------------------
\input{titlepage}
\newpage

%------------------
% Table of Contents
%------------------
% \thispagestyle{empty}

\tableofcontents 
% some thesis documents wil also want a table of contents for figures and tables
% uncomment the commands below to create a table of contents for figures and tables
%\listoffigures 
%\listoftables
 
\newpage
 
% page numbering can be roman (e.g. i, ii, iii, iv), alpha (a, b, c, d)
% or arabic (1,2,3,4). Change the below option to roman for front matter text
% such as preface pages, alph for alpha lettering (e.g. Appendix maybe)
% or arabic for standard page numbering. If you want to restart the page 
% counter because you are using a new numbering format, you can use 
% \setcounter{page}{X} where X is the new number you want to count on.
\pagenumbering{arabic}

%-------------------------------------------------------------------------------------------------------------%
% Abstract
%-------------------------------------------------------------------------------------------------------------%
\subfile{../sections/abstract}

%-------------------------------------------------------------------------------------------------------------%
% Objective
%-------------------------------------------------------------------------------------------------------------%
\subfile{../sections/objective}

%-------------------------------------------------------------------------------------------------------------%
% Background
%-------------------------------------------------------------------------------------------------------------%
\subfile{../sections/background}

%-------------------------------------------------------------------------------------------------------------%
% Research Design & Methods
%-------------------------------------------------------------------------------------------------------------%
\subfile{../sections/methods}

%-------------------------------------------------------------------------------------------------------------%
% Preliminary Results
%-------------------------------------------------------------------------------------------------------------%
\subfile{../sections/results}

%-------------------------------------------------------------------------------------------------------------%
% Example code - not references
%-------------------------------------------------------------------------------------------------------------%
% \subfile{../sections/example_code}

This is a random citation \cite{nrtagu2001arsenic,lessler1988lead}

%---------------------------------------------------------------------------------------------%
% Bibliography
%---------------------------------------------------------------------------------------------%
\newpage

%-----------------------%
% automatic bib entries %
%-----------------------%
% enter your bibliographies using a .bib file
% for most formats, the unsrt argument (unsorted) will list the bibliographies as cited in the 
% text, rather than sorting them alphabetically. For the \bibliography entries, much like the 
% \graphicspath entries, each relative file directory is separated by a comma, no space,
% followed by the .bib file name, without the .bib extension
\bibliographystyle{unsrt} 
\bibliography{../bib/bib_example}
% example: \bibliography{../bib/bib_example,../bib/bib_example2,../bib/bib_example3}

%-----------------------------%
% manual (easier) bib entries %
%-----------------------------%
% you can manually insert bibliography entries using thebibliography enviornment
% note the argument {99} just represents that you expect at least 2 digits worth of 
% bibliographies. For example, 11 and 99 represent the same amount, and 101 represents
% 3 digits worth (at most 999 entries) of bibliographies

% \begin{thebibliography}{99} 

% to add a bibliography manually, use \bibitem{ID} with ID being a unique identifier to that 
% entry. The ID can be used to \cite{ID} to make a reference. Details below \bibitem is the 
% bibliography itself. It will not be formmatted and will show as is. The details, in whatever
% format are best taken from Google Scholar

% \bibitem{bitem}
% this is a bibitem

% \end{thebibliography}

%---------------------------------------------------------------------------------------------%
% Appendix
%---------------------------------------------------------------------------------------------%
\newpage
\appendix
\subfile{../sections/appendix}

%---------------------------------------------------------------------------%
%								 End Document								%
%---------------------------------------------------------------------------%
\end{document}